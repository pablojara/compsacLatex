\documentclass[conference]{IEEEtran}
\IEEEoverridecommandlockouts
% The preceding line is only needed to identify funding in the first footnote. If that is unneeded, please comment it out.
\usepackage{cite}

\usepackage{latexsym}% para los graficos
\usepackage{epsfig}

\usepackage{amsmath,amssymb,amsfonts}
\usepackage{graphicx}
\usepackage{textcomp}
\usepackage{xcolor}
\usepackage{amsmath}
\usepackage{algorithm}
\usepackage[T1]{fontenc}
\usepackage[noend]{algpseudocode}
\newcommand{\algmargin}{\the\ALG@thistlm}
\makeatother
\newlength{\whilewidth}
\settowidth{\whilewidth}{\algorithmicwhile\ }
\algdef{SE}[parWHILE]{parWhile}{EndparWhile}[1]
  {\parbox[t]{\dimexpr\linewidth-\algmargin}{%
     \hangindent\whilewidth\strut\algorithmicwhile\ #1\ \algorithmicdo\strut}}{\algorithmicend\ \algorithmicwhile}%
\algnewcommand{\parState}[1]{\State%
  \parbox[t]{\dimexpr\linewidth-\algmargin}{\strut #1\strut}}
\algdef{SE}[parIF]{parIf}{EndParIf}[1]
  {\parbox[t]{\dimexpr\linewidth-\algmargin}{%
     \hangindent\whilewidth\strut\algorithmicwhile\ #1\ \algorithmicdo\strut}}{\algorithmicend\ \algorithmicwhile}%

\makeatletter
\def\BState{\State\hskip-\ALG@thistlm}
\makeatother


\def\BibTeX{{\rm B\kern-.05em{\sc i\kern-.025em b}\kern-.08em
    T\kern-.1667em\lower.7ex\hbox{E}\kern-.125emX}}
\begin{document}

\title{Network awareness for  scheduling of containerized applications in an multilevel edge/fog/cloud distributed environment*\\
\thanks{This work has been funded by UNED through the Independent Thinking-J\'ovenes Investigadores 2019 project efFIcient scheduLing of containErs (FILE, 2019-PUNED-0007), by Ministry of Economy and Competitiveness (Ministerio de Econom\'ia y Competitividad, MINECO) through the Red Tem\'atica Espa\~nola de Anal\'itica de Aprendizaje, Spanish Network of Learning Analytics (SNOLA, RED2018-102725-T), and by the Region of Madrid through the project e-Madrid-CM (P2018/TCS-4307).
}
}

\author{\IEEEauthorblockN{Pablo Jaramillo, Agust\'in C. Caminero, Roc\'io Mu\~noz}
\IEEEauthorblockA{\textit{National Distance University of Spain (UNED)} \\
Madrid, Spain \\
pjaramill7@alumno.uned.es, accaminero@scc.uned.es, rmunoz@dia.uned.es}
}

% EATA: Emerging Advances in Technology & Applications

\maketitle

\begin{abstract}


The use of containers to deploy applications has suffered a huge growth in the last years. Containers allow the easy deployment of applications in such a way they can be isolated from the underlying hardware without the resource consumption and the performance degradation of traditional virtualization technologies (e.g. virtual machines). With this regard, containers can be used in a variety of scenarios, some of them harness distributed multilevel infrastructures (edge, fog and cloud). In these scenarios, the network plays an essential role in the proper execution of the applications, since it is the communication medium for all the agents in such distributed scenario and its poor performance may lead to poor quality of service of the entire system. This paper presents a new technique to perform scheduling of containerized applications on distributed multilevel infrastructures  (edge, fog, and cloud). This technique considers the status of the interconnection network as a top-class resource to be considered when performing the scheduling tasks. A performance evaluation using a cluster of microcomputers Raspberry  Pi has been conducted as a proof of concept of or work.



\end{abstract}

\begin{IEEEkeywords}
scheduling algorithm, network, virtualization, container, distributed system, edge, fog, cloud 
\end{IEEEkeywords}



%%%%%%%%%%%%%%%%%%%%%%%%%%%%%%%%%%%%%%%%%%%%
\section{Introduction and motivation}
%%%%%%%%%%%%%%%%%%%%%%%%%%%%%%%%%%%%%%%%%%%%

During the last years, distributed heterogeneous computing resources have arisen. \emph{Cloud computing} can be considered as the application of business models existing in traditional utilities (such as water or power supply) to computing equipment, where users consume virtualized storage or compute resources offered by a provider based on a pay-as-you-go model. The resources providers use are large datacenters with plenty of computing, memory and storage capacity. Along with this, devices at the \emph{edge} of the network can be  part of Internet of Things (IoT) deployments.  These can be sensors and devices that are part of Smart City infrastructure, lifestyle gadgets like wearables, and smart appliances or smart phones. These devices have been originally designed to sense and generate data streams, and they have some spare capacity (computing, memory, and storage) which could be harnessed to execute applications~\cite{GhoshKS18}. 

The last term to be coined is \emph{fog computing}, which refers to computing resources that are between the edge and cloud in the network hierarchy, with compute, storage and memory capacities that fall between these layers as well~\cite{Simmhan19}. These edge and fog resources provide the opportunity for low latency processing of the generated data, closer to its source, and on the wide-area network~\cite{Stojmenovic2014}. As a result, there is critical need to understand how this diverse ecosystem of edge, fog and cloud resources can be effectively used by large-scale distributed applications~\cite{Varshney2019}. This multilevel scenario can be seen in Figure~\ref{fig:edgefogcloud}~\cite{BittencourtMBRP17}, where the term \emph{cloudlets}  refers to the access points devices extended with computing and storage services to create the fog system.
 
In such multilevel environment, the network plays an essential role since it is the communication medium between all the actors in the system and its poor performance affects the whole system. Thus, the network should be taken into account when making all kinds of decisions in order to provide Quality of Service (QoS), one of the main ones is scheduling. The term \emph{scheduling} refers to the process of allocating computing resources to an application and mapping constituent components of that application onto those resources, in order to meet certain QoS and resource conservation goals~\cite{Pinedo08}. % The term \emph{application} can be seen as an abstract entity created using different primitives such as processes, threads, or workflows, among others~\cite{Tucker89, Taylor07}. In the same way, a \emph{computing resource} can be very diverse, including among others bare metal server, a cluster of physical or virtual servers locally or in the cloud, edge or mobile devices in an IoT deployment~\cite{LiuYDA19, Zhan2015}. Regarding \emph{QoS}, it can be latency, packet loss, throughput, among many other metrics, and \emph{conservation goals} can be among others the energy footprint of the system. So, scheduling applications requires understanding these terms and the way how they coexist. 

\begin{figure}[t!]
\begin{center}
\strut\psfig{figure=Imagenes/edgefogcloud.eps,width=.5\textwidth}
\caption{Multilevel edge/fog/cloud scenario~\cite{BittencourtMBRP17}.}\label{fig:edgefogcloud}
\end{center}
\end{figure}


Virtualization technologies allow the package and execution of applications in isolated environments, independently from the underlying hardware resources. The most traditional virtualization technology is \emph{hardware virtualization}, which creates \emph{virtual machines (VM)}, which are abstractions of hardware on top of which an operating system runs. They require extensive hardware resources to run since VMs run full operating systems. A more recent way of virtualization is \emph{operating system-level virtualization}, which creates \emph{containers}~\cite{Buyya13}. They  use the host operating system and thus they do not require one for each container (as for the VMs), so the hardware requirements to run containers are lower than for VMs. Thanks to this, containers are suitable to run on low cost computers such as Raspberry Pi.  

In order to deploy containers, this can be done in a cluster of computers. In this case, there is a need to manage such cluster, and this is done by the \emph{container orchestrator}, one of the best known being Kubernetes~\cite{Bernstein14b}. Among the tasks the orchestrator has to perform, one of the most important is the scheduling of the containers into the nodes of the cluster.

%As commented above, the network in a distributed system is a first class resource that should be taken into account in order to provide QoS. In a cluster, an overloaded network may make the containers not provide the requested QoS. In order to tackle this issue, 

Considering all of this, this work proposes an scheduling algorithm in which the container orchestrator uses the status of the network, along with more information such as the memory or CPU status of the nodes, in order to choose the node of the cluster that will run a container -- this is the main contribution of this work. 
%Thus, the contributions of this paper are as follows. First, a network-aware algorithm to schedule containers in a cluster is presented, this is the main contribution. 
Also, this paper presents proof-of-concept implementation of the algorithm on Kubernetes, and a performance evaluation of the algorithm, which has been conducted by deploying Kubernetes on a cluster of microcomputers Raspberry Pi~\cite{VujovicM15}. Raspberry Pi has been chosen because of its low cost and its ability to run docker containers. Kubernetes has been chosen because it has been used for compute containers on Raspberry Pi-class edge devices~\cite{Tsai17}, and used to deploy software on the fly on fog resources~\cite{Hong16}. 

This paper is organized as follows. Section~\ref{sec:rel} presents the related work of this paper. Section~\ref{sec:alg} presents the scheduling algorithm which is the main contribution of this work. Section~\ref{sec:eval} details the performance evaluation conducted to illustrate the usefulness of our work. Finally, Section~\ref{sec:conc} concludes the paper and suggests guidelines for future work. 

%%%%%%%%%%%%%%%%%%%%%%%%%%%%%%%%%%%%%%%%%%%%
\section{Related work}\label{sec:rel}
%%%%%%%%%%%%%%%%%%%%%%%%%%%%%%%%%%%%%%%%%%%%

The growth of the use of light virtualization technologies or containers is evident in the vast majority of areas where previously the hardware virtualization was used. Most of the times, the deployment of an application is done in a distributed computing environment, in the cloud or in a local cluster. When deploying applications in distributed environments a container orchestration tool like Kubernetes is needed. Kubernetes manages the deployment of the application introducing the \emph{pod} concept which is the basic deployment unit in the system. A pod is capable of running one or more containers and every pod has to be deployed in 
a selected node of the distributed computing environment which will be selected by Kubernetes based on the resources required by the node to run smoothly and
in some affinities and taints which has to be configured depending on the type of deployment. Studies have been conducted that address the effects of the deployment 
configuration in the performance and robustness of the application. In [5] a performance model of Kubernetes application is obtained with the application of benchmarks.
It defends that for applications with more than eight containers, the bandwidth and final throughput is higher if every container is deployed in a separate pod. This suggest
that deploying more pods with fewer containers in each pod, is better in terms of performance and bandwidth availability. In [6], from the same authors, the 
effects of the type of deployment explained above are studied and detailed depending on the nature of the application and its services. The paper makes a
distinction between high CPU and I/O demanding services and high bandwidth demand. At the same time, it makes a distinction between service container,
that are executing full time, and job containers, which are created and destroyed with each execution. 
For CPU and I / O intensive applications it is argued that if there is more containers than nodes in the cluster, the best option is to group all the 
containers, however, it is important to divide the pods among all possible nodes, so it is recommended to divide the number of containers by the number of nodes 
available in the cluster and deploy a pod on each node. If there is a smaller number of containers than nodes, a container in each pod and a pod in each node. 
For the second type of application, intensive use of network resources, it is recommended the same policy applied previously in the case of type containers
service, while for job-type containers it is always recommended to deploy a single container per pod. This is due to the fact that the network overload in a pod
increases significantly when several containers are executed, since they all share the same IP address, unique in each pod.

These studies are very useful when making decisions about the deployment of an application on infrastructures with container orchestration.
Despite all this, more precise adjustments may be required in the process of assignment of a node to each pod. If the application is based on a large number of
job-type containers that are created and destroyed proportionally with the load of work, it may be the case that the default Kubernetes planner is not
enough. This scheduler takes into account resources such as CPU and memory, but it does not take into account the network congestion at the time of assignment 
or the overhead of the interfaces During this work, a Kubernetes and Raspberry Pi based testing environment will be deployed and a container planning algorithm
based on resource metrics per node and network congestion will be developed.



%%%%%%%%%%%%%%%%%%%%%%%%%%%%%%%%%%%%%%%%%%%%
\section{Container scheduling Algorithm Development}\label{sec:alg}
%%%%%%%%%%%%%%%%%%%%%%%%%%%%%%%%%%%%%%%%%%%%



The Kubernetes scheduling algorithm, from now on, scheduler, has the only mission of scheduling
the just created pods to a cluster node. It is a process executing with the rest of components
in the master node. When a new pod is created, an event is created ant the Kubernetes scheduler is 
triggered to start the assignation of the pod. It has three main execution phases: filtering,
node ranking and node selection. During the filtering phase, the scheduler discards the nodes
that are not suitable to run the pod due to restrictions and taints. During the second phase, 
the scheduler populates a list containing all the suitable nodes for the pod execution, applying
a collection of priority functions to every node, trying to divide the pods among all the
available nodes and giving more importance to the ones with less workload. During the last execution
phase, the node with higher priority is selected, and in case of existing more than one, it a random 
one is choosen. 
The scheduler is written in Golang, just like the rest of the Kubernetes project. The scheduler
performance is a key part of the system, and in some cases it become the bottleneck of the cluster.
For this work, the developed algorithm will only implement the last two phases of the original
scheduling algorithm, avoiding the filtering phase as it is out of scope in the work.


% Detalles sobre el algoritmo. Explicar qué es nodemetrics, bestbandwidthnode, bestRecPackets... y todas las variables que se utilizan en el algoritmo. 


\begin{algorithm}

\caption{Custom Scheduler}\label{Kubernetes}
\begin{algorithmic}[1]
\Function{calculatePriorities}{}
\State Let $nodeMetrics[i].cpu$ be blablabla
\State Let $bestRecPackets$ be blablabla

\For{\textit{int i = 0; i < 4; i++}}
\If {$\textit nodeMetrics[i].cpu < bestCpu$}
\State $\textit bestCpu \gets nodeMetrics[i].cpu[i]$
\State $\textit bestCpuNode \gets nodeMetrics[i].name$
\EndIf
\If {$\textit nodeMetrics[i].memory < bestMemory$}
\State $\textit bestMemory \gets nodeMetrics[i].memory[i]$
\State $\textit bestMemoryNode \gets nodeMetrics[i].name$
\EndIf
\If {$\textit nodeMetrics[i].recPackets < bestRecPackets$}
\State $\textit bestRecPackets \gets nodeMetrics[i].recPackets[i]$
\State $\textit bestRecPacketsNode \gets nodeMetrics[i].name$
\EndIf
\If {$\textit nodeMetrics[i].sentPackets < bestSentPackets$}
\State $\textit bestSentPackets \gets nodeMetrics[i].sentPackets[i]$
\State $\textit bestSentPacketsNode \gets nodeMetrics[i].name$
\EndIf
\If {$\textit nodeMetrics[i].bandwidth > bestBandwidth$}
\State $\textit bestBandwidth \gets nodeMetrics[i].bandwidth[i]$
\State $\textit bestBandwidthNode \gets nodeMetrics[i].name$
\EndIf
\If {$\textit nodeMetrics[i].diskIO < bestDiskIO$}
\State $\textit bestDiskIO \gets nodeMetrics[i].diskIO[i]$
\State $\textit bestDiskIONode \gets nodeMetrics[i].name$
\EndIf
\EndFor
\BState
\State $\textit nodePriorities[bestCpuNode]+=3$
\State $\textit nodePriorities[bestMemoryNode]+=2$
\State $\textit nodePriorities[bestRecPacketsNode]+=1$
\State $\textit nodePriorities[bestSentPacketsNode]+=1$
\State $\textit nodePriorities[bestBandwidthNode]+=2$
\State $\textit nodePriorities[bestDiskIONode]+=1$
\BState
\State \textbf{return} $nodePriorities$
\EndFunction
\end{algorithmic}

\end{algorithm}



The scheduler that has been developed makes use of the metrics that Prometheus exports on each node through the node exporters.
This means that requests to know the status of each node will be made to each of the node exporters since it is desired to know
the status at that same moment. If they were made to the Prometheus Server, to obtain the measurements you would have to make the 
requests to the exporter node, so they will be made directly from the scheduler saving an extra jump when passing through the server.
These metrics obtained from each export node are used to calculate the priority of the candidate nodes to be assigned to the pod
in planning. During the planning process, five metrics exported by Prometheus are taken into account: current frequency of each core
of the node's CPU, percentage of memory occupied, read and write operations on disk and the number of packets sent and received by the
user interface. network in use. The ideal node for the assignment with the pod is the one that uses the least amount of all these
resources at the time of selection. After this, the iPerf metrics are collected and stored into the scheduler container, to feed
the scheduling algoruthm among the Prometheus metrics.

For the node ranking the scheduling algorithm will compare each node metrics, giving, for every metric, a pondered score to the node
with best metrics. Each metric score has a different value, being CPU, memory and bandwidth the ones with highest weight.
The priority computation function implements this logic, and is executed for every new pod in the cluster.
The pseudocode for this logic can be seen in the figure below.

The scheduler code is compiled and packaged in a container and deployed in a pod, and has to be allowed with a role based access account to schedule
and allocate new pods.


%%%%%%%%%%%%%%%%%%%%%%%%%%%%%%%%%%%%%%%%%%%%
\section{Performance Evaluation}\label{sec:eval}
%%%%%%%%%%%%%%%%%%%%%%%%%%%%%%%%%%%%%%%%%%%%

In order to illustrate the proposed algorithm, a performance evaluation has been conducted. This is detailed in this section, where the experiments setup is explained in the fist place, followed by an analysis of the results obtained. 


%%%%%%%%%%%%%%%%%%%%%%%%%%%%%%%%%%%%%%%%%%%%
\subsection{Work Environment Preparation}\label{sec:env}
%%%%%%%%%%%%%%%%%%%%%%%%%%%%%%%%%%%%%%%%%%%%


\textbf{Detalles: Poner foto del cluster de raspberrys, incluir referencias bien puestas con bibtex (en el fichero BIBLIOGRAFIA.bib), pantallazos del prometheus, graficas de resultados}

For the testing environment, four Raspberry Pi 3 Model B and one AMD64 Intel Core i7 architecture computer have been used. They have been connected using a five ports ethernet switch. A Kubernetes cluster has been deployed over this hardware infrestructure configuring the AMD computer as the master node and
the Raspberries as worker nodes, and tainting the master node to avoid the execution of pods on it. Prometheus Operator[10] 
has also been installed in the cluster, so we are able to monitor the cluster resources usage, CPU and memory occupation, pods distribution and network metrics.
iPerf tool [12] which allows network monitoring between two or more nodes has also been installed and configured in the cluster, to collect information and metrics
about the available bandwidth between each node, packet loss and delay. With this two monitoring tools we are able to know the state of the cluster and feed the
container scheduling algorithm which will select the best node for a certain pod taking all this data in account.



%%%%%%%%%%%%%%%%%%%%%%%%%%%%%%%%%%%%%%%%%%%%
\subsection{Experiments and results}\label{sec:expe}
%%%%%%%%%%%%%%%%%%%%%%%%%%%%%%%%%%%%%%%%%%%%



Performance is really important for an scheduling algorithm.


%%%%%%%%%%%%%%%%%%%%%%%%%%%%%%%%%%%%%%%%%%%%
\section{Conclusions and future work}\label{sec:conc}
%%%%%%%%%%%%%%%%%%%%%%%%%%%%%%%%%%%%%%%%%%%%

Distributed systems spanning infrastructures at different levels (edge/fog/cloud) have gained popularity in the recent years, and an appropriate scheduling technique is needed in order to provide QoS to the running applications. 



%\begin{thebibliography}{00}
%\bibitem{b1} G. Eason, B. Noble, and I. N. Sneddon, ``On certain integrals of Lipschitz-Hankel type involving products of Bessel functions,'' Phil. Trans. Roy. Soc. London, vol. A247, pp. 529--551, April 1955.
%\bibitem{b2} J. Clerk Maxwell, A Treatise on Electricity and Magnetism, 3rd ed., vol. 2. Oxford: Clarendon, 1892, pp.68--73.
%\bibitem{b3} I. S. Jacobs and C. P. Bean, ``Fine particles, thin films and exchange anisotropy,'' in Magnetism, vol. III, G. T. Rado and H. Suhl, Eds. New York: Academic, 1963, pp. 271--350.
%\bibitem{b4} K. Elissa, ``Title of paper if known,'' unpublished.
%\bibitem{b5} R. Nicole, ``Title of paper with only first word capitalized,'' J. Name Stand. Abbrev., in press.
%\bibitem{b6} Y. Yorozu, M. Hirano, K. Oka, and Y. Tagawa, ``Electron spectroscopy studies on magneto-optical media and plastic substrate interface,'' IEEE Transl. J. Magn. Japan, vol. 2, pp. 740--741, August 1987 [Digests 9th Annual Conf. Magnetics Japan, p. 301, 1982].
%\bibitem{b7} M. Young, The Technical Writer's Handbook. Mill Valley, CA: University Science, 1989.
%\end{thebibliography}
%\vspace{12pt}
%\color{red}
%IEEE conference templates contain guidance text for composing and formatting conference papers. Please ensure that all template text is removed from your conference paper prior to submission to the conference. Failure to remove the template text from your paper may result in your paper not being published.


\bibliographystyle{IEEEtran}
% argument is your BibTeX string definitions and bibliography database(s)
% argument is your BibTeX string definitions and bibliography database(s)
% \bibliography{IEEEabrv,BIBLIOGRAFIA.bib}
\bibliography{BIBLIOGRAFIA}


\end{document}
